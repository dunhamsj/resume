\documentclass[10pt]{article}
\usepackage[letterpaper,tmargin=1in,rmargin=1in,bmargin=1in,lmargin=1in]%
           {geometry}
\usepackage[utf8]{inputenc} % Not sure what this does
\usepackage{url}
\usepackage{hyperref}
\hypersetup{
  bookmarksnumbered = true,
  bookmarksopen=false,
  pdfborder=0 0 0,   % make all links invisible
                     %   so the pdf looks good when printed
  pdffitwindow=true, % window fit to page when opened
  pdfnewwindow=true, % links in new window
  colorlinks=true,   % false: boxed links; true: colored links
  linkcolor=blue,    % color of internal links
  citecolor=magenta, % color of links to bibliography
  filecolor=magenta, % color of file links
  urlcolor=blue     % color of external links
}

\begin{document}

\flushleft

{\large \textbf{Samuel John Dunham}}
\vspace{1em}

Email: \href{mailto:samueljdunham@gmail.com}%
                   {samueljdunham@gmail.com}
\vspace{0.5em}

Personal Website: \url{https://www.samueljdunham.com}
\vspace{0.5em}

GitHub: \url{https://www.github.com/dunhamsj}

\vspace{1em}

% Education
\textbf{Education}\vspace{0.5em}\hrule\vspace{1em}

\textbf{Vanderbilt University}, Nashville TN\newline
\textit{Ph.D. Astronomy \& Astrophysics}, May 2024 (expected)
\vspace{1em}

% Work Experience
\textbf{Work Experience}\vspace{0.5em}\hrule\vspace{1em}

\textbf{Vanderbilt University}, Nashville TN\newline
\textit{Research Assistant}, 08/2016 - Present
\begin{itemize}\setlength\itemsep{0.1cm}
  \item
    Writing, implementing, and testing
    Fortran90 code to solve general relativistic hydrodynamics
    equations with high-order discontinuous Galerkin methods
    in \href{https://www.github.com/endeve/thornado}{\texttt{thornado}}
    for execution on exascale supercomputers
  \item
    Porting code to run on GPUs via OpenACC and OpenMP Offloading
  \item
    Coupling code to \href{https://www.github.com/AMReX-Codes/amrex}{AMReX}
    for MPI parallelism and block-structured adaptive mesh refinement
\end{itemize}
\vspace{1em}

\textbf{University of Michigan}, Ann Arbor MI\newline
\textit{Research Assistant}, 06/2014 - 05/2016
\begin{itemize}\setlength\itemsep{0.1cm}
  \item
    Analyzed Hubble Space Telescope data using \texttt{SAOImageDS9}
    for multiple images of
    background sources due to strong gravitational lensing by galaxy clusters
  \item
    Using \texttt{LENSTOOL},
    found robust lens models for several galaxy clusters,
    from which was deduced the mass of the cluster core,
    the total magnification provided by the cluster,
    the location of the source, and its morphology
  \item
    One of these models led to a publication in the Astrophysical Journal:
    Samuel J. Dunham \textit{et al} 2019 \textit{ApJ} \textbf{875} 18
    (\url{https://iopscience.iop.org/article/10.3847/1538-4357/ab0d7d})
\end{itemize}
\vspace{1em}

% Skills
\textbf{Skills}\vspace{0.5em}\hrule

\begin{itemize}\setlength\itemsep{0.1cm}
  \item
    Complex problem solving
  \item
    Interpersonal communication
  \item
    Proficient in (neo)vim, Bash, git/github, Python, and Fortran90
  \item
    Familiar with Julia and C++
  \item
    Debugging code
  \item
    Submitting and monitoring jobs
    on leadership-class supercomputers using tools like \texttt{Slurm}
    and \texttt{PBS}
  \item
    Optimizing software for GPUs via OpenACC and OpenMP offloading
  \item
    LaTeX
\end{itemize}

\end{document}
